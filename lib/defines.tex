\makeatletter
\newcommand{\DeclareLatinAbbrev}[2]{%
  \DeclareRobustCommand{#1}{%
    \@ifnextchar{.}{\textit{#2}}{%
      \@ifnextchar{,}{\textit{#2.}}{%
        \@ifnextchar{!}{\textit{#2.}}{%
          \@ifnextchar{?}{\textit{#2.}}{%
            \@ifnextchar{)}{\textit{#2.}}{%
              {\textit{#2.,\ }}}}}}}}%
}
\newcommand{\DeclareNonLatinAbbrev}[2]{%
  \DeclareRobustCommand{#1}{%
    \@ifnextchar{.}{\textit{#2}}{%
      \@ifnextchar{,}{\textit{#2.}}{%
        \@ifnextchar{!}{\textit{#2.}}{%
          \@ifnextchar{?}{\textit{#2.}}{%
            \@ifnextchar{)}{\textit{#2.}}{%
              {\textit{#2.\ }}}}}}}}%
}
\makeatother
\DeclareLatinAbbrev{\eg}{e.g}
\DeclareLatinAbbrev{\Eg}{E.g}
\DeclareLatinAbbrev{\ie}{i.e}
\DeclareLatinAbbrev{\Ie}{I.e}
\DeclareLatinAbbrev{\etc}{etc}
\DeclareNonLatinAbbrev{\etal}{et~al}

\def\first {$(i)$\xspace}
\def\Second{$(ii)$\xspace}
\def\third {$(iii)$\xspace}
\def\fourth{$(iv)$\xspace}
\def\fifth {$(v)$\xspace}
\def\sixth {$(vi)$\xspace}
\def\seventh{$(vii)$\xspace}
\def\eighth{$(viii)$\xspace}

%%% custom definitions
%% Coloring the links!
\newcommand\myshade{75} % Usage: red!\myshade!black

\definecolor{ForestGreen} {RGB}{34,  139,  34}
\definecolor{HeraldRed2}   {rgb}{0.81, 0.12, 0.15}

\newcommand{\refscolor} {blue}
\newcommand{\linkscolor}{HeraldRed2}
\newcommand{\urlscolor} {ForestGreen}


% For runin headings
\newcommand{\smartparagraph}[1]{\vspace{.05in}\noindent\textbf{#1}}

%% Table of Contents (ToC) depth 
\setcounter{secnumdepth}{4} % how many sectioning levels to assign numbers to
\setcounter{tocdepth}{4}    % how many sectioning levels to show in ToC

%% Limit hyphenation
\hyphenpenalty=9000
\tolerance=5000
% Reduce hyphenation as much as possible:
%\hyphenpenalty=15000
%\tolerance=1000


% For notes by the authors to themselves
\newcommand*{\todoinline}[1]{\textcolor{red}{TODO: #1}}
% https://tex.stackexchange.com/questions/411566/tikz-how-to-modify-pie-chart-with-individually-adjustable-colors-and-point-numb
%\DeclareUnicodeCharacter{2003}{\quad}
\newcommand\MemoryLayout[1]{
  \begin{tikzpicture}[scale=0.3]
     \draw[thick](0,0)--++(0,3)node[above]{$0$};
     \foreach [parse=true] \pt/\col/\lab [remember=\pt as \tp (initially 0)] in {#1} {
       \foreach \a in {\tp,...,\pt-1} {
          \draw[fill=\col](-\a,0) rectangle ++(-1,2);
       }
       \draw[thick](-\pt,0)--++(0,3)node[above]{$\pt$};
       \if\lab\relax\relax\else
         \draw[thick,decorate, decoration={brace,amplitude=4mm}]
            (-\tp,-0.2)--node[below=4mm]{\lab} (-\pt,-0.2);
       \fi
     }
  \end{tikzpicture}
}
\newcommand{\slice}[6]{
  \pgfmathparse{0.5*#1+0.5*#2}
  \let\midangle\pgfmathresult
  % \show\midangle
  % slice
%  \draw[thick, pattern=#5, pattern color=#6] (0,0) -- (#1:1) arc (#1:#2:1) -- cycle;
  \draw[thick, pattern=#5, pattern color=#6] (0,0) -- (#1:1) arc (#1:#2:1) -- cycle;
  % outer label
  \node[label=\midangle:#4] at (\midangle:1) {};

  % inner label
  \pgfmathparse{min((#2-#1-10)/110*(-0.3),0)}
  \let\temp\pgfmathresult
  \pgfmathparse{max(\temp,-0.5) + 0.8}
  \let\innerpos\pgfmathresult
  \node[fill=white, inner sep=1pt, outer sep=0pt, text=white, text opacity=0] at (\midangle:\innerpos) {#3};
  \node at (\midangle:\innerpos) {#3};
}

% facilitates the creation of memory maps. Start address at the bottom, end address at the top.
% Addresses will be print with a leading '0x' and in upper case.
% syntax: \memsection{end address}{start address}{height in lines}{text in box}

% facilitates the creation of memory maps. Start address at the bottom, end address at the top.
% Addresses will be print with a leading '0x' and in upper case.
% syntax: \memsection{end address}{start address}{height in lines}{text in box}
\newcommand{\memsection}[4]{
	\bytefieldsetup{bitheight=#3\baselineskip}	% define the height of the memsection
	\bitbox[]{8}{
		\texttt{0x\uppercase{#1}}	 % print end address
		\\ \vspace{#3\baselineskip} \vspace{-2\baselineskip} \vspace{-#3pt} % do some spacing
		\texttt{0x\uppercase{#2}} % print start address
	}
	\bitbox{16}{#4} % print box with caption
}


\renewcommand\tikzmark[1]{%
  \tikz[remember picture,overlay]\node (#1) {};%
}
\newcommand\Connect[3][]{%
\tikz[remember picture,overlay]
  \draw[->,red,>=latex,#1] (#2.north east) -- ( $ (#3.north west) + (-2pt,0) $ );%
}
%https://tex.stackexchange.com/questions/324008/patterned-legend-in-caption
\newcommand{\legendsquare}[1]{%
  \textcolor{#1}{\rule{1ex}{1ex}}%
}
%https://tex.stackexchange.com/questions/225740/how-to-make-a-custom-legend-with-a-squared-box-followed-by-a-text
\newcommand{\cbox}[1]{\raisebox{\depth}{\fcolorbox{black}{#1}{\null}}}