%%% Local Variables:
%%% mode: latex
%%% TeX-master: t
%%% End:
% The following command is used with glossaries-extra
\setabbreviationstyle[acronym]{long-short}
% The form of the entries in this file is \newacronym{label}{acronym}{phrase}
%                                      or \newacronym[options]{label}{acronym}{phrase}
% see "User Manual for glossaries.sty" for the  details about the options, one example is shown below
\newacronym{ABI}{ABI}{Application Binary Interface}

\newacronym{API}{API}{Application Programming Interface}
% note the specification of the long form plural in the line below
\newacronym[longplural={Debugging Information Entities}]{DIE}{DIE}{Debugging Information Entity}
%
% The following example also uses options
\newacronym[shortplural={OSes}, firstplural={operating systems (OSes)}]{OS}{OS}{operating system}
%\newacronym{ASLR}{ASLR}{Address Space Layout Randomization}
\newacronym[shortplural={CLA}, firstplural={Cross Language Attacks(s)}]{CLA}{CLA}{Cross Language Attack(s)}
\newabbreviation{CVE}{CVE}{Common Vulnerabilities and Exposures}
\newabbreviation{CWE}{CWE}{Common Weakness Enumeration}
\newacronym{FDA}{FDA}{United States Food and Drug Administration}
\newacronym{FFI}{FFI}{Foreign Function Interface}
% note the use of a non-breaking dash in long text for the following acronym
\newacronym{IQL}{IQL}{Independent Q‑Learning}
\newacronym{IoT}{IoT}{Internet of Things}
\newacronym{JDK}{JDK}{Java development Kit }
\newacronym{JNI}{JNI}{Java Native Interface}
\newacronym{KTH}{KTH}{KTH Royal Institute of Technology}
\newacronym{LAN}{LAN}{Local Area Network}
\newacronym{LOC}{LOC}{lines of code}
\newacronym{NSA}{NSA}{United States of America, National Security Agency}
\newacronym{VM}{VM}{Virtual Machine}
% note the use of a non-breaking dash in the following acronym
\newacronym{WiFi}{Wi‑Fi}{Wireless Fidelity}
\newacronym{WLAN}{WLAN}{Wireless Local Area Network}
%\newacronym{UN}{UN}{United Nations}
%\newacronym{SDG}{SDG}{Sustainable Development Goal}

